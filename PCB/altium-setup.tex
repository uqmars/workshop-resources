\documentclass[a4paper,12pt]{report}
\usepackage[utf8]{inputenc}
\usepackage[sfdefault]{roboto}

\usepackage{titling}
\usepackage{graphicx}
\usepackage{wrapfig}
\usepackage{subcaption}
\usepackage{float}
\usepackage{fontawesome}
\usepackage{setspace}
\usepackage[export]{adjustbox}
\usepackage[margin=20mm]{geometry}
\usepackage{xcolor}
\usepackage{hyperref}
\hypersetup{
    colorlinks=true,
    linkcolor=blue,
    filecolor=magenta,      
    urlcolor=blue,
}
\urlstyle{same}

\definecolor{turbo_purple}{RGB}{112,105,160}

\usepackage{titlesec}
\titleformat{\section}[block]{\normalfont\huge\bfseries\centering}{}{1em}{}
\titleformat{\subsection}[display]{\normalfont\Large\bfseries\color{turbo_purple}}{}{1em}{}
\titleformat{\subsubsection}[display]{\normalfont\large\bfseries}{}{}{\vspace{-1.2em}}[\vspace{-0.9em}]

\usepackage{fancyhdr}
\pagestyle{fancy}
\usepackage{tikz}
\usetikzlibrary{calc}
\usepackage{tikzpagenodes}
\fancyfoot{}
\renewcommand{\headrulewidth}{0pt}
\setlength{\headheight}{25pt}
\rhead{\begin{tikzpicture}[remember picture,overlay]
\draw  let \p1=($(current page.north)-(current page header area.south)$),
      \n1={veclen(\x1,\y1)} in
node [inner sep=0,outer sep=0,below left] 
      at (current page.north east){\includegraphics[height=\n1]{PCB/Assests/Sponsorship Header - Solid.png}};
\end{tikzpicture}}
\lfoot{\begin{tikzpicture}[remember picture,overlay]
\draw  let \p1=($(current page footer area.north)-(current page.south)$),
      \n1={veclen(\x1,\y1)} in
node [inner sep=0,outer sep=0,above right] 
      at (current page.south west){\includegraphics[height=\n1]{PCB/Assests/Sponsorship Footer - Solid.png}};
\end{tikzpicture}}
\cfoot{\sffamily\selectfont\thepage}

\begin{document}

\begin{titlepage}
    \newgeometry{right=0mm,left=0mm,top=20mm,bottom=0mm}
    \begin{center}
        \vspace*{15mm}
        \includegraphics[width=0.4\paperwidth]{PCB/Assests/Logo (Dark).png} \\
        \vspace{1cm}
        \includegraphics[width=0.25\paperwidth]{PCB/Assests/EBESS_logo-04.png} \\
        \vspace{1cm}
        \Huge Altium Designer Set Up Instructions \\
        \huge \textcolor{turbo_purple}{2023}
    \end{center}
    \vfill
    \includegraphics[height=0.5\paperheight, right]{PCB/Assests/Pattern - PCB (Solid).png}
    \vspace*{10mm}
\end{titlepage}
\restoregeometry

\section*{Preparing for the Workshop}

Before the workshop, please ensure you try follow one of the following to gain access to and utilise Altium on your computer. 
\textbf{You will need to bring your laptop to this workshop}
The sections have been ordered from easiest to hardest methods of obtaining/using Altium on your system.

\subsection*{Remote Desktop Protocol (Recommended)}

This method is essential if you have a M1 Mac, a low-spec laptop, or have low storage.

\begin{enumerate}
    \item Follow the instructions provided by UQ EAIT \href{https://student.eait.uq.edu.au/infrastructure/remote-access/rdp.html}{here}.
    \item Locate Altium on this and open the software
\end{enumerate}

If you cannot open Altium on your login through this method. You may not have been given a licence by UQ. Please attempt the instructions in \textit{Installation Via Altium Education}.


\subsection*{Installation and Connection Via VPN}

If you would like to access the software directly on your laptop, follow the instructions provided by UQ EAIT \href{https://student.eait.uq.edu.au/software/altium/}{here}. Please note \textbf{we will not help you install this on the night}.


\subsection*{Installation Via Altium Education}

This is the final method of getting Altium on your system. Because it can take some time to verify, if you plan to use this method, please request immediately.

\begin{enumerate}
    \item Follow \href{https://www.altium.com/education/student-licenses}{this link}
    \item Register with your student email sXXXXXXX@student.uq.edu.au
    \item Respond to the activation email
    \item Download and install Altium Designer
\end{enumerate}


\end{document}
