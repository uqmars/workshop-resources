\documentclass[a4paper,12pt]{report}
\usepackage[utf8]{inputenc}
\usepackage[sfdefault]{roboto}
\usepackage{mdframed}
\usepackage{tabularx}
\usepackage{multirow}

\usepackage{titling}
\usepackage{graphicx}
\usepackage{wrapfig}
\usepackage{subcaption}
\usepackage{float}
\usepackage{fontawesome}
\usepackage{setspace}
\usepackage[export]{adjustbox}
\usepackage[margin=20mm]{geometry}

\usepackage{xcolor}
\definecolor{turbo_purple}{RGB}{112,105,160}

\usepackage{titlesec}
\titleformat{\section}[block]{\normalfont\huge\bfseries\centering}{}{1em}{}
\titleformat{\subsection}[display]{\normalfont\Large\bfseries\color{turbo_purple}}{}{1em}{}
\titleformat{\subsubsection}[display]{\normalfont\large\bfseries}{}{}{\vspace{-1.2em}}[\vspace{-0.9em}]

\usepackage{fancyhdr}
\pagestyle{fancy}
\usepackage{tikz}
\usetikzlibrary{calc}
\usepackage{tikzpagenodes}
\fancyfoot{}
\renewcommand{\headrulewidth}{0pt}
\setlength{\headheight}{25pt}
\rhead{\begin{tikzpicture}[remember picture,overlay]
\draw  let \p1=($(current page.north)-(current page header area.south)$),
      \n1={veclen(\x1,\y1)} in
node [inner sep=0,outer sep=0,below left] 
      at (current page.north east){\includegraphics[height=\n1]{Assets/Sponsorship Header - Solid.png}};
\end{tikzpicture}}
\lfoot{\begin{tikzpicture}[remember picture,overlay]
\draw  let \p1=($(current page footer area.north)-(current page.south)$),
      \n1={veclen(\x1,\y1)} in
node [inner sep=0,outer sep=0,above right] 
      at (current page.south west){\includegraphics[height=\n1]{Assets/Sponsorship Footer - Solid.png}};
\end{tikzpicture}}
\cfoot{\sffamily\selectfont\thepage}

\usepackage[hidelinks]{hyperref}
\hypersetup{colorlinks=false}

\newcommand{\imageWidth}{140mm}
\newcommand{\diagramWidth}{150mm}
\newcommand{\codeHeight}{7mm}

\begin{document}

\begin{titlepage}
    \newgeometry{right=0mm,left=0mm,top=20mm,bottom=0mm}
    \begin{center}
        \vspace*{15mm}
        \includegraphics[width=0.7\paperwidth]{Assets/Logo (Dark).png} \\
        \vspace{1cm}
        \Huge Microcontroller Workshop \\
        \huge \textcolor{turbo_purple}{2023}
    \end{center}
    \vfill
    \includegraphics[height=0.5\paperheight, right]{Assets/Pattern - PCB (Solid).png}
    \vspace*{10mm}
\end{titlepage}
\restoregeometry


\section*{Thank you to our partners!}
\begin{figure}[!htbp]
    \centering
    \caption*{Turbo Sponsors}%
    \subfloat{{\includegraphics[width=7.5cm]{Assets/BPEQ-Logo-Web.png} }}%
    \qquad
    \subfloat{{\includegraphics[width=7.5cm]{Assets/black_corematic_logo_inline_expertises.png} }}%
    \qquad
    \subfloat{{\includegraphics[width=7.5cm]{Assets/JAY-0002-Brand-Identity-Evolution-Logo_Full-Colour-RGB.png} }}%
\end{figure}

\begin{figure}[!htbp]
    \centering
    \caption*{Gold Sponsors}%
    \subfloat{{\includegraphics[width=5.5cm]{Assets/ACS-RCL_Logo-Colour.png} }}%
    \qquad
    \subfloat{{\includegraphics[width=5.5cm]{Assets/chelona.png} }}%
\end{figure}

\begin{figure}[!htbp]
    \centering
    \caption*{Silver Sponsors}%
    \subfloat{{\includegraphics[width=3cm]{Assets/EA_logo_Flat_CMYK.png} }}%
\end{figure}

% \pagebreak
\newpage

\section*{Required Software}
The required software for these tutorials is Arduino IDE. To download the software, go through the following steps:
    \begin{enumerate}
        \item Head to the software page on the Arduino website and click on the appropriate installer. Follow the prompts and install on your computer
        
        \begin{center}
            \includegraphics[width = \imageWidth]{Assets/Downloads_page.png}
        \end{center}
        
        \item Next, open the IDE and head to \textbf{File → Preferences}, and enter the following URL into the 'Additional boards manager URLs':
        \textbf{https://raw.githubusercontent.com/espressif/arduino-esp32/gh-pages/package\_esp32\_index.json}
        
        \begin{center}
            \includegraphics[width = \imageWidth]{Assets/Esp_url.png}
        \end{center}
        
        \newpage
        \item Go back to the IDE and head to \textbf{Tools → Board → Boards Manager}. In the old IDE, a new pop-up should appear, and in IDE 2, the toolbox should appear on the left hands side as shown below:
        
        \begin{center}
            \includegraphics[width = \imageWidth]{Assets/Boards_manager.png}
        \end{center}
    
        \item To upload code to a board, plug it into the computer and make sure the board is chosen and the port specified (\textbf{Tools → Port → 'Select correct port'}). 
        Then, hit the right-facing arrow to upload. Remember to hold down the boot button on the ESP32 while uploading.
        \begin{center}
            \includegraphics[width = \imageWidth]{Assets/Upload.png}
        \end{center}
        For the ESP32 uploading, once the message below is seen, release the boot button and press the reset button.
        \begin{center}
            \includegraphics[scale = 0.7]{Assets/Hard_reset.png}
        \end{center}
    \end{enumerate}


\newpage

\section*{What Is A Microcontroller}

    A microcontroller is a small computer on a single chip that controls some or all of the 
    functions of an electronic device or system. It has a processor, memory and input/output 
    peripherals. Microcontrollers are used in many devices, such as cars, robots, home appliances
    and medical devices. \\


    Microcontrollers usually have features such as:

    \begin{itemize}
        \item A central processing unit (CPU) that performs arithmetic, logic and I/O operations
        \item Volatile memory (RAM) for data storage
        \item Non-volatile memory (ROM, EPROM, EEPROM or Flash) for program and operating parameter storage
        \item Discrete input and output bits that allow control or detection of individual pins
        \item Serial input/output ports (UARTs) that enable communication with other devices
        \item Other peripherals such as timers, counters, analog-to-digital converters, etc.
    \end{itemize}
    

\section*{Hobby Microcontroller Options}
    \begin{itemize}
        \item Arduino Family
        \begin{itemize}
            \item Arduino Uno (ATmega328P)
            \item Arduino Mega (ATmega2560)
            \item Ardiono Nano (ATmega328P)
        \end{itemize}
        \item Raspberry Pi Family 
        \begin{itemize}
            \item Raspberry Pi Model B
            \item Raspberry Pi Zero
            \item Raspberry Pi Pico
        \end{itemize}
        \item Espressif Family
        \begin{itemize}
            \item ESP8266 Development Boards
            \item ESP32 Development Boards
        \end{itemize}
    \end{itemize}

\newpage

\section*{Powering A Microcontroller} {

    \subsection*{Arduino Input Voltage Range}
        The voltage range of an Arduino board depends on the specific board model. 
        However, most Arduino boards operate on a voltage range of 5V to 12V DC. 
        Some newer boards, such as the Arduino Due, can operate on a wider range of voltages, from 7V to 20V DC.
        Your laptop USB ports usually supply a well regulated logic-level voltage of 5V, so this 

    \subsection*{Main Power Supply Voltage vs Logic-Level Power}
        Arduino boards have two power input pins: the Vin pin and the 5V pin. 
        The Vin pin is connected to the main power supply voltage, which can be in the range of 
        5V to 12V or higher, depending on the board. The 5V pin, on the other hand, is connected 
        to the logic-level power supply, which is usually 5V or 3.3V.

        The main power supply voltage is used to power external components, such as sensors, 
        motors, and other devices connected to the Arduino board. 
        The logic-level power supply, on the other hand, is used to power the 
        microcontroller and other digital components on the board.

    \subsection*{Drivers and Shields}
        Drivers and shields are add-on boards that are designed to extend the functionality of an 
        Arduino board. Drivers are used to control motors, relays, and other high-power devices that 
        require more current than the Arduino board can provide. Shields, on the other hand, are add-on 
        boards that provide additional features, such as Wi-Fi connectivity, GPS, and LCD displays.
        
        Both drivers and shields may require a separate power supply, depending on the power requirements 
        of the devices they are controlling. In some cases, they may also require level shifting circuitry 
        to ensure that the signals sent between the Arduino board and the driver or shield are compatible. \\
        
        
        \vspace{5mm}
        \begin{itemize}
            \item Sheilds Available at Jaycar: \\
                \url{https://www.jaycar.com.au/search?text=sheild}
            \item Motor Shield (Usefull for ENGG1100): \\
            \url{https://www.jaycar.com.au/arduino-compatible-motor-servo-controller-module/p/XC4472}
    
        \end{itemize}
        
        \vspace{1cm}
        \emph{(These items are usually significantly cheaper online - but JayCar is good in a pinch!)}
        
}
\newpage

\section*{Overview of the Arduino and ESP32}
    \begin{center}
    
    Arduinos and ESPs are types of microcontrollers that can control simple to complex electronics. At a basic fundamental level, they are able to control other electrical devices, and read incoming data.
    In this tutorial, purple boxes indicate new knowledge on component or code.
    \vspace{10mm}
    \includegraphics[scale = 0.7]{Assets/Arduino_overview.png}

    \begin{itemize}
        \item Indicated above in green, is the data cable connection for
        uploading codes from the computer. Connecting this also provides power to the board
        \item Indicated in yellow, are the digital ports. These ports are able to interpret and send signals in binary. That is, they sense or output either ON or OFF
        \begin{itemize}
            \item Note that the ports with a tilde ($\sim$) are able to give pwm outputs.
        \end{itemize}
        \item Indicated in red are the analog ports. These ports are able to interpret signals in a range. For example, they can read a range of 0 - 255.
        \begin{itemize}
            \item Note that analog ports may also be used as digital ports but not vice versa. Also, analog ports may only read in analog, they cannot write in pwm
        \end{itemize}
        \item GND - These ports are used as a negative terminal
        \item 5V/3.3V - These are the voltage supply ports
    \end{itemize}
    \end{center}

\newpage
\begin{center}
    \includegraphics[scale = 2]{Assets/Esp_pinout.jpg}
    \begin{itemize}
        \item The pins in green are the GPIO pins which can serve as inputs and outputs unless otherwise signed
        \item Ports labelled ADC can be used to read analog signals
        \item Ports labelled TOUCH have internal capacitive touch sensors. For example, these pins can be used as touch sensors using just a finger
        \item All pins that can use output can use PWM
    \end{itemize}
\end{center}

\newpage

\section*{Lesson 1: Blinking an LED}
    \subsection*{What You'll Need:}
        \begin{itemize}
            \item Arduino/ESP32
            \item LED
            \item Breadboard
        \end{itemize}
        \begin{mdframed}[linewidth = 3, linecolor = turbo_purple]
        \textbf{LED:}
        \\
        The new component is a light emitting diode (LED). It is a low-current light source that operates with currents only in a single direction. The longer pin is the positive terminal.
        \includegraphics[scale = 0.3]{Assets/led.png}
        \end{mdframed}
    \subsection*{Wiring Diagram:}
        \begin{center}
            \includegraphics[width = \diagramWidth]{Assets/LED_diagram.png}
        \end{center}
   \subsection*{Directions:}
        \begin{center}
            \begin{tabularx}{\textwidth} { 
                | >{\raggedright\arraybackslash}X 
                | >{\raggedright\arraybackslash}X | }
                \hline
                \multicolumn{1}{|c|}{\textbf{Arduino}} & 
                \multicolumn{1}{|c|}{\textbf{ESP32}} \\
                \hline
                \begin{enumerate}
                    \item Plug the led with positive terminal in port 13, and negative terminal in ground
                \end{enumerate} &
                \begin{enumerate}
                    \item Plug the led with positive terminal in port G23, and negative terminal in ground
                \end{enumerate} \\
                \hline
            \end{tabularx}
        \end{center}

\newpage
\subsection*{Code (Light the LED):}
    \begin{mdframed}[linewidth = 3, linecolor = turbo_purple]
        \includegraphics[height = \codeHeight]{Assets/int.png} \\
        This line creates a new integer variable of any name and sets it to a specified value. A variable name can be anything (note capitals are identified as different characters). A variable is a sort of container that stores information in it. \\ \\
        \includegraphics[height = \codeHeight]{Assets/pinmode.png} \\
        This line assigns a chosen variable as an output or an input on the Arduino. The Arduino takes the information from the variable and uses it as a port address. \\ \\
        \includegraphics[height = \codeHeight]{Assets/dwrite.png} \\
        This line digitally powers the chosen port to either high or low,  meaning current on or off.
    \end{mdframed} 
    \vspace{20mm}
    \includegraphics[width = \diagramWidth]{Assets/Light_led.png} \\
    Upload to the Arduino/ESP32 and the LED should light up.

\newpage
\subsection*{Code (Blinking the LED):}
    \begin{mdframed}[linewidth = 3, linecolor = turbo_purple]
        \includegraphics[height = \codeHeight]{Assets/delay.png} \\
        Pauses code operation for a period of time (milliseconds). The Arduino/ESP32 will continue its previous given operations during this period.
    \end{mdframed}
    \vspace{10mm}
    \includegraphics[width = \diagramWidth]{Assets/blink_led.png} \\
    Upload the code to the Arduino/ESP32 and the LED should blink. Changing the value of the delay will change the frequency of the blinking.

\newpage

\section*{Lesson 2: LED with a button}
    \subsection*{What You'll Need:}
        \begin{itemize}
            \item Arduino/ESP32
            \item LED
            \item Breadboard
            \item 2 wires (male to male)
            \item Button
        \end{itemize}
        \begin{mdframed}[linewidth = 3, linecolor = turbo_purple]
            \textbf{Breadboard:} \\
            A breadboard is a non-soldering technique of connecting components. Wires in the same column are connected, and some breadboards have indicated rows connected horizontally as shown below. \\
            \includegraphics[scale = 0.7]{Assets/breadboard.png} \\
            \textbf{Button:} \\
            A button is a simple method of breaking a circuit. Standard buttons have 2 pins however, some buttons have 4. Indicated below, pressing a button connects 1 \& 3 to 2 \& 4. \\
            \includegraphics[scale = 1]{Assets/button.jpg}
        \end{mdframed}
\newpage
\subsection*{Wiring Diagram:}
    \begin{center}
        \includegraphics[width = \diagramWidth]{Assets/button_led.png}
    \end{center}

\subsection*{Directions:}
\begin{center}
    \begin{tabularx}{\textwidth} { 
        | >{\raggedright\arraybackslash}X 
        | >{\raggedright\arraybackslash}X | }
        \hline
        \multicolumn{1}{|c|}{\textbf{Arduino}} & 
        \multicolumn{1}{|c|}{\textbf{ESP32}} \\
        \hline
        \begin{enumerate}
            \item Plug the positive terminal of the Arduino in port 13, and negative to GND. Connect the button to the breadboard as indicated.
            \item Using wires, connect the top left terminal to port A0, and connect the bottom right terminal to GND
        \end{enumerate} &
        \begin{enumerate}
            \item Plug the positive terminal of the ESP32 in G23, and negative to GND. Connect the button to the breadboard as indicated.
            \item Using wires, connect the top left terminal to GND, and connect the bottom right terminal to G12
        \end{enumerate} \\
        \hline
    \end{tabularx}
\end{center}

\subsection*{Code:}
    \begin{mdframed}[linewidth = 3, linecolor = turbo_purple]
        \includegraphics[height = \codeHeight]{Assets/dread.png} \\
        Reads the incoming digital signal from a sensor. \\
        \includegraphics[height = 21mm]{Assets/if.png} \\
        Logic statements are used to give the boards intelligence. For example, if a variable is equal to a certain number, the microcontroller will run a specific command for those conditions. Operators such as the following may be used: \\
        \includegraphics[scale = 0.4]{Assets/operators.png}
    \end{mdframed}
    \newpage
    \begin{center}
        \includegraphics[width = \diagramWidth]{Assets/led_button.png}\\
    \end{center}
    Upload the code to the Arduino/ESP32. When the button is not pressed, the LED should be off, and when the button is pressed, the LED should turn on.

\newpage

\section*{Lesson 3: PWM Servo Motor Control}
    \subsection*{What You'll Need:}
            \begin{itemize}
                \item Arduino/ESP32
                \item Servo Motor
                \item LED \& Pushbutton
                \item Breadboard
                \item 5 wires (male to male)
            \end{itemize}
        \begin{mdframed}[linewidth = 3, linecolor = turbo_purple]
            \textbf{Servo Motor:}
            \\
            A servo motor is a type of motor that can control its position, speed and torque precisely 
            with feedback from a sensor and a controller. It is used for applications that require accurate 
            and dynamic movements, such as robotics, CNC machinery and automated manufacturing. A servo motor 
            consists of a suitable motor, a sensor for position feedback and a controller that interprets the 
            input signal and adjusts the motor accordingly.            
            The servo motors used in this workshop are controlled using a PWM signal.
            \\
            \includegraphics[scale = 1]{Assets/servos.png}
        \end{mdframed}
        \newpage
        \begin{mdframed}[linewidth = 3, linecolor = turbo_purple]
            \textbf{Pulse Width Modulation (PWM):}
            \\
            PWM: Pulse width modululation takes a digital signal (high or low) and rapidly switches the signal 
            on and off. This technique is used to control the amount of power delivered to a load 
            (Can be thought of as analog power). The duration of the on and off periods can determine 
            the average power delivered to the load. This is creates a square wave signal, this 
            signal can be described by its duty cycle and its period.  Where duty cycle is how 
            much of the signal is on during one single period. 
            \\
            \includegraphics[scale = 1]{Assets/pwm1.jpg}
        \end{mdframed}
    \subsection*{Wiring Diagram:}
        \includegraphics[width = \diagramWidth]{Assets/servoWiring.png}
    \newpage
    \subsection*{Directions:}
        \begin{center}
            \begin{tabularx}{\textwidth} { 
                | >{\raggedright\arraybackslash}X 
                | >{\raggedright\arraybackslash}X | }
                \hline
                \multicolumn{1}{|c|}{\textbf{Arduino}} & 
                \multicolumn{1}{|c|}{\textbf{ESP32}} \\
                \hline
                \begin{enumerate}
                    \item Instructions pending...
                \end{enumerate}  &
                \begin{enumerate}
                    \item Connect the servos red wire to 5V, black wire to GND and the yellow wire to pin 12.
                    \item Need to import ESP32Servo.h, this library will need to be downloaded from the ‘Library Manager’
                    \item (servo).setPeriodHertz this sets the frequency at which the pwm signal occurs
                    \item (servo).attach changes the corresponding pin to output a pwm signal
                    \item (servo).write writes an angle for the PWM 
                    \item More useful functions can be found through reading the documentation:
                            \url{https://madhephaestus.github.io/ESP32Servo/classServo.html#a613aa8f5aa81971357e6ff36df92309d}
                    
                \end{enumerate} \\
                \hline
            \end{tabularx}
            
        \end{center}
    \newpage
    \subsection*{Code (ESP32):}
        \begin{mdframed}[linewidth = 3, linecolor = turbo_purple]
            \includegraphics[width = \diagramWidth]{Assets/servoCode.png}
        \end{mdframed}
    \newpage

\newpage

\section*{Lesson 4: Wireless Communication}

    Wireless communication is essential for remotely operated devices/robotics.
    Although not required, you are usually encouraged to use wireless 
    communication in the ENGG1100 group projects. Wireless control may also come in handy in further
    group projects such as ENGG2800 and METR2800. \\

    Alternatively, many groups will also just use a long USB / serial cable to remotely operate their 
    machine - yuck. After this tutorial you will hopefully gain the confidence to use 
    wireless communication in your future projects and enjoy the benefits of a more flexible and reliable system. \\

    For this lession we will obviously need a microcontroller with wireless cababilities. UQ MARS has supplied a limited number
    of ESP32's. These are like cool little arduinos, but are faster, cheaper, have more memory and most importantly feature
    WiFi (2.4 GHz) and bluetooth capabilites. Alternatively, with a couple of tweaks, you can also use an ESP8266 with this tutorial 
    - ask one of our execs if you have any trouble. 
    \textbf{Share one of the ESP32's with a buddy if there isn't enough to go around}.\\

    FYI: JayCar sells ESP32's and various arduino WiFi products which use the ESP8266.


    \subsection*{Overview}
    We will need to set up a wireless communication link between your Arduino board and your computer.
    This link is called a "TCP Socket" and it can be thought of as a 'virtual usb port'. A socket is identified using both an IP Address and a port number.
     you can read more about them here: 
    \url{https://www.howtouselinux.com/post/tcp-socket}. \\

    In this lesson, we will configure the ESP32 to host its own wireless network (you can think of it like your phone's hotspot).
    You will then connect your laptop to the ESP32's wireless network. Now that both devices are
    on the same network - we can connect them using a socket (a virtual USB cable - if you like).\\

    
    We will need to set up a "socket" on both ends of the connection:
    \begin{itemize}
        \item ESP32 - "Server" Socket
        \item Computer (python script) - "Client" Socket
    \end{itemize}

    \newpage
    \subsection*{Code (ESP32 - Server):}
        \begin{mdframed}[linewidth = 3, linecolor = turbo_purple]
            \includegraphics[width = \diagramWidth]{Assets/servoCode.png}
        \end{mdframed}

    \subsection*{Code (Python - Client):}
        \begin{mdframed}[linewidth = 3, linecolor = turbo_purple]
            \includegraphics[width = \diagramWidth]{Assets/servoCode.png}
        \end{mdframed}



    \newpage

% \section*{Advanced: Program Microcontroller with C}

% https://docs.espressif.com/projects/esp-idf/en/latest/esp32/

% \newpage



\section*{References}

        \begin{itemize}
            \item (Paraphrase from OpenAI's ChatGPT AI language model, personal communication, March 25, 2023)
        \end{itemize}


\section*{Credits}

        \begin{itemize}
            \item Binara Wasala
            \begin{itemize}
                \item "Required Software" section
                \item "Overview of the Arduino and ESP32" section
                \item "Lesson 1: Blinking an LED" section
                \item "Lesson 2: LED with a button" section
            \end{itemize}
            \item Reuben Richardson
            \begin{itemize}
                \item "What Is A Microcontroller" \& "Hobby Microcontroller Options" sections
                \item "Lesson 3: PWM Servo Motor Control" section
                \item "Lesson 4: Wireless Communication" section
            \end{itemize}
            \item Hokul Srikantthan
            \begin{itemize}
                \item "Lesson 3: PWM Servo Motor Control" section
                \item "Powering A Microcontroller" section
            \end{itemize}
        \end{itemize}


\end{document}
